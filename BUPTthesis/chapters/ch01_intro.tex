% 学位论文 : 第一章  绪论
% 
% 更新记录:
%   {$LastChangedBy$}
%   {$LastChangedRevision$}
%   {$LastChangedDate$}

\chapter{绪论}

\section{研究背景}


随着社会的发展与科学技术的进步,人们对信息服务的需求量与日俱增,信息产业已经成为关系国计民生的支柱性产业。世界各国纷纷投入大量的资金、人力、物力,致力于发展本国的信息产业,美国“信息高速公路”,韩国“IT大运河”,欧盟“宽带战略”等等,信息产业已经成为体现一个国家综合国力的重要指标。为了适应全球信息产业革命的进程,《国家中长期科学和技术发展规划纲要》明确提出:到2020 年,我国要“掌握一批事关国家竞争力的装备制造业和信息产业核心技术,使制造业和信息产业技术水平进入世界先进行列。”作为现代信息技术的主要载体,光传输网络的发展在我国信息产业建设中起着举足轻重的作用。所以,光网络的发展就成为我国信息产业建设的首要任务。同时,这种战略上的重大需求也给人们带来了各种技术上的挑战。如何建设功能更强、可靠性更好,功耗及成本更低、体积更小、使用维护更方便的光通信网络,成为信息技术研究人员所关注的热点。

显然,为了实现以上研究目标,单纯地依靠网络层面的优化是远远不够的,基础器件层面的进展更能够带动光通信网络的重大发展。目前,光通信器件还大多是相互独立,依靠传统的技术手段将其连接,组合成光网络。为了进一步降低成本,减小器件体积,光电子集成(即光电子器件之间,光电子与微电子器件之间的集成)就成为光通信器件的大势所趋。我们知道,制备光电子器件(如激光器,探测器,放大器等等)的基本材料都是半导体材料,尤其是Ⅲ-Ⅴ族半导体材料,为了实现不同器件之间的集成,不同的半导体材料之间的异质兼容就成为一个重要的研究方向。目前,光电子器件主要是GaAs基与InP基器件,微电子器件则是以Si基为主,而且,Si基造价低廉,应用广泛,且对于1.3-1.6μm的通信波长范围,Si是透明的,所以, Si基与GaAs基器件的异质兼容,一方面对光电子集成具有重大意义,另一方面能够有效地降低器件的制作成本。

自1963年阿尔费洛夫和克罗默两位科学家提出了半导体双异质结构以来,光电子器件经过了将近半个世纪的发展,已经取得了丰硕的成果,为光通信产业的快速发展起到了不可替代的推动作用。然而,相对于目前已经发展成熟的Si基大规模集成电路而言,光电子器件的集成度还不可同日而语。因此,将Si与III-V族材料集成成为了长期以来一个重要的研究方向。


\section{技术难点}

\section{研究现状}
我来占个位置。\cite{BUPT_Thesis_Format_2004}


% 本章参考文献
\ifx\usechapbib\empty
\bibliographystyle{buptthesis}
\bibliography{bare_thesis}
\fi

%%% Local Variables: 
%%% mode: latex
%%% TeX-master: "bare_thesis"
%%% End: 
