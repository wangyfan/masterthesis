% 学位论文 : 第一章  绪论
% 
% 更新记录:
%   {$LastChangedBy$}
%   {$LastChangedRevision$}
%   {$LastChangedDate$}

\chapter{绪论}

\section{研究背景}

Si是电子学最基本的材料。大约95\%的半导体器件是用Si衬底制作的。作为一个载体,Si衬底由于其质量轻、良好的热传导性、价格低、晶片半径大和易获取使其无疑有很大的优势。另一方面,在Si之后的电子材料——III-V族化合物,尤其是GaAs——是光电子基本的材料,并且III-V族材料的高载流子流动率形成了其制作高速特殊用途器件的基础。因此,将Si与III-V族材料集成成为了长期以来一个重要的研究方向。

向这个目标迈进的第一步,便是获得Si衬底上高质量的GaAs薄层,形成所谓的人工衬底。


\section{技术难点}

\section{研究现状}

% 本章参考文献
\ifx\usechapbib\empty
\bibliographystyle{buptthesis}
\bibliography{../bare_thesis}
\fi

%%% Local Variables: 
%%% mode: latex
%%% TeX-master: "bare_thesis"
%%% End: 
