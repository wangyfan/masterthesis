% 
% 更新记录:
%   {$LastChangedBy$}
%   {$LastChangedRevision$}
%   {$LastChangedDate$}

% 涉密论文保密年限
\classdur{三年}
        
% 学号
\studentid{2011111735}

% 论文题目
\ctitle{Si基GaAs异变外延及Si基异变量子点的研究}

% 申请学位
\cdegree{工学硕士}

% 院系名称
\cdepartment{信息光子学与光通信研究院}

% 专业名称
\cmajor{电子科学与技术}

% 你的姓名
\cauthor{王一帆} 

% 你导师的姓名
\csupervisor{任晓敏}

% 日期自动生成,也可以取消注释下面一行,自行指定日期
\cdate{\CJKdigits{2013}年\CJKnumber{12}月}

% 中文摘要
\cabstract{%
  本文介绍北京邮电大学研究生学位论文~\LaTeX{}~模板的使用方法。%
  本模板基本符合北京邮电大学研究生院培养与学位办公室于~2004~年~1~月~6~%
  日颁布的《北京邮电大学关于研究生学位论文各式的统一要求》。%

  中、英文摘要位于声明的次页,摘要应简明表达学位论文的内容要点,体现研%
  究工作的核心思想。重点说明本项科研的目的和意义、研究方法、研究成果、%
  结论,注意突出具有创新性的成果和新见解的部分。%

  关键词是为文献标引工作而从论文中选取出来的、用以表示全文主题内容信息%
  的术语。关键词排列在摘要内容的左下方,具体关键词之间以均匀间隔分开排%
  列,无需其它符号。%
}

% 中文关键词,关键词之间用\kwsep分割
\ckeywords{\TeX \kwsep \LaTeX \kwsep CJK \kwsep 模板 \kwsep 排版 \kwsep论文}

% 英文摘要
\eabstract{%
  This article presents, the \LaTeX{} thesis template for doctor/master 
  thesis of Beijing University of Posts and Telecommunications, and briefly 
  introduces the usage. The template fulfills the corresponding format 
  requirements issused by academic administration of Beijing University of 
  Posts and Telecommunications Graduate School on January 6, 2004. 

  The Chinese and English abstract should appear after the declaration page. 
  The abstract should present the core of the research work, especially the 
  purpose and importance of the research, the method adopted, the results, 
  and the conclusion.

  Key words are terms selected for documentation indexing, which should 
  present the main contributions of the thesis. Key words are aligned at the 
  bottom left side of the abstract content. Key words should be seperated by 
  spaces but not any other symbols.
}

% 英文关键词,也用\kwsep分割
\ekeywords{%
  \TeX \kwsep \LaTeX \kwsep CJK \kwsep template \kwsep 
  typesetting \kwsep thesis}
%\makeatother

%%% Local Variables:
%%% mode: latex
%%% TeX-master: "bare_thesis"
%%% End:
